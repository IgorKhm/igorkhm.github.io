\documentclass[a4paper,11pt]{exam}
%\printanswers % pour imprimer les réponses (corrigé)
 \noprintanswers % Pour ne pas imprimer les réponses (nonc)
% \addpoints % Pour compter les points
% \noaddpoints % pour ne pas compter les points
\qformat{\textbf{Exercice \thequestion \,:}\\} % Questions style
\usepackage{color} % Define new colors

\usepackage[utf8x]{inputenc}
\usepackage[T1]{fontenc}
\usepackage{exercise}
\usepackage{mathrsfs,amsmath,amssymb,latexsym,amsfonts,mathtools,stmaryrd}
\usepackage{enumitem}
\setitemize{label=\textbullet}
% Bigger binomial in math mode
\usepackage{nccmath}
\renewcommand{\binom}{\mbinom}

\usepackage[francais]{babel}

\usepackage{tikz}
\usetikzlibrary{trees}

\shadedsolutions % définit le style des réponses
% \framedsolutions % définit le style des réponses 
\definecolor{SolutionColor}{rgb}{0.8,0.9,1} % bleu ciel
\renewcommand{\solutiontitle}{
	\noindent\textbf{Solution:}\par\noindent} % Définit le titre des réponses
\newcommand\eqdef[0]{\stackrel{\text{def}}{=}}

\pagestyle{headandfoot}
\headrule
\header{L3 - 2019/2020 - TD6}{Wednesday 23 October}{Mathématiques discrètes}
\footer{S. Le Roux, I. Khmelnitsky}{\thepage}{E.N.S. Cachan}
\footrule

\renewcommand{\questionshook}{%
	\setlength{\labelwidth}{0pt}%
	\setlength{\itemsep}{0.9\baselineskip}
}

\definecolor{gris}{gray}{0.95}
\newcommand{\Z}{\mathbb{Z}}
\newcommand{\N}{\mathbb{N}}
\newcommand{\Q}{\mathbb{Q}}
\newcommand{\R}{\mathbb{R}}
\newcommand{\C}{\mathbb{C}}
\newcommand{\F}{\mathbb{F}}
\newcommand{\A}{\mathcal{A}}
\renewcommand{\S}{\mathcal{S}}
\newcommand{\RR}{\mathbin{\mathcal{R}}}
\begin{document}
	
	
	\begin{questions}		
		
		\qformat{\textbf{Exercice \thequestion~(\thequestiontitle):}\\}
		\titledquestion{Dilworth’s theorem}
		Recall Dilworth’s theorem~: 
		Let $k$ be the maximal cardinality of an antichain in $E$.
		Then $E$ is a disjoint union of $k$ chains(a set of comparable elements).		
		\begin{enumerate}			
			
			\item Let  $\mathcal{I}$ be a family of $N\in \N$ closed intervals in $\R$. Let $m$ and $n$ be two natural numbers such that $N-1=nm$.
			Show that there are either $m+1$ disjoint intervals in $\mathcal{I}$ or there are $n+1$ intervals with a non empty intersection.
			
			% Soit $N$ un entier naturel non nul. Soit $\mathcal{I}$ une famille de
			%   $N$ intervalles fermés réels.
			%   Soit $m$ et $n$ deux entiers naturels tels que $N-1=nm$.
			%   Démontrer qu'il existe $m+1$ intervalles dans $\mathcal{I}$ disjoints
			%   deux à deux ou $n+1$ intervalles dans $\mathcal{I}$ d'intersection non
			%   vide.
			\newcommand{\I}{\mathcal{I}}
			
			\begin{solution}
				We define a partial order $\preceq$ on the familly $\I$ follows~:
				\[
				[a,b] \prec [c,d] \text{ if } b<c
				\]
				The chains of $(\I,\preceq)$ are are intervals $[a_1,b_1], \dots, [a_k,b_k]$ s.t $a_1 < b_1 < a_2 < b_2 < \cdots < b_{k-1} < a_k < b_k$.
				The antichains are intervals $[a_1,b_1], \dots, [a_k,b_k]$ s.t. $[a_i,b_i] \cap [a_j,b_j]\neq \emptyset$ for any $i<j\in[k]$. 
				If there doesn't exists a family of intervals which don't intersect each other of size $m+1$ then by Dilworth's theorem there is an antichain of size $>n$ and we are done. We pick $a = \max(a_1, \dots, a_k)=a_j$ and $b = \max(b_1, \dots, b_k)=b_l$. Since for any $i$ the intersection $[a_i,b_i] \cap [a_j,b_j]$ is not empty, hence $a_i < a < b_i$, and $[a_i,b_i] \cap [a_l,b_l]$ is not empty, and hence $a_i < b < b_i$.
				Therefore $[a,b]$ is a non empty interval contained in each $[a_i,b_i]$.
				
				%taking let $[a,b]$ be the intersection of $[a_1,b_1]$ and $[a_2,b_2]$. For any $i\notin\{1,2\}$  we get that $[a_i,b_i]$ 
				
				
				
				
				% Les chaines sont les suites d'intervalles $[a_1,b_1], \dots, [a_k,b_k]$
				% avec $a_1 < b_1 < a_2 < b_2 < \cdots < b_{k-1} < a_k < b_k$.
				% Les antichaines sont les suites d'intervalles
				% $[a_1,b_1], \dots, [a_k,b_k]$ telles que $[a_i,b_i] \cap [a_j,b_j]$
				% soit vide pour tous $i,j$.
				% S'il n'existe pas $m+1$ intervalles dans $\I$ disjoints deux à deux,
				% les chaines sont toutes de cardinal $\leq m$.
				% En utilisant le théorème de Dilworth, $\I$ est de cardinal $\leq mk$ où
				% $k$ est le maximum des cardinaux des antichaines dans $\I$.
				% Donc $k>n$.
				% Soit alors $[a_1,b_1], \dots, [a_k,b_k]$ une antichaine de cardinal
				% $k$.
				% On pose $a = \max(a_1, \dots, a_k)=a_j$ et
				% $b = \max(b_1, \dots, b_k)=b_l$.
				% Alors pour tout $i$, $[a_i,b_i] \cap [a_j,b_j]$ est non vide, donc
				% $a_i < a < b_i$ (en particulier $a < b$), et
				% $[a_i,b_i] \cap [a_l,b_l]$ est non vide, donc $a_i < b < b_i$.
				% Donc $[a,b]$ est un intervalle non vide contenu dans tous les
				% $[a_i,b_i]$.
			\end{solution}
		\end{enumerate}
		
		\qformat{\textbf{Exercice \thequestion~(\thequestiontitle):}\\}
		\titledquestion{Théorème de Cantor-Bernstein}
		Let $A$ and $B$ two sets and $f:A \to B$ and $g: B \to A$ two injective functions.
		Let $H:\mathcal{P}(A)\to  \mathcal{P}(A)$ the map~:
		$$
		X \mapsto A \setminus g[B \setminus f[X]]
		$$
		\begin{enumerate}
			\item Using the Knaster-Tarski theorem, show that $ H $ has a fixed point
			
			\begin{solution}
				Since $(\mathcal{P}(A),\subseteq)$ is complete lattice we are left with showing that $H$ is order preserving: 
				\begin{align*}
				E \subseteq&  F\\
				H(E) \subseteq& H(F)\\
				B\setminus H(E) \supseteq&  B\setminus H(F)\\
				g\left(B\setminus H(E)\right)  \supseteq&  g\left(B\setminus H(F)\right)\\
				A\setminus g\left(B\setminus H(E)\right) \subseteq& A\setminus g\left(B\setminus H(F)\right)
				\end{align*}
				Since a complete lattice has cannot be empty we are done.
				
			\end{solution}
			
			\item Deduce that  $A$ and $B$ are equipotent.
			
			\begin{solution}
				Let $Z$ be a fixed point of $H$, i.e. $Z = A \setminus g[B \setminus f[Z]]$, hence $A \setminus Z = g[B \setminus f[Z]]$.
				Note that, if $x\in A\setminus Z$ then $x\in g(B)$, if $x \in g[B]$ it has a unique preimage $g^{-1}(x)$, since $g$ is injective.				
				We define a map $h$ as follows:
				\begin{align*}
				h : & A \to B\\
				& x \mapsto
				\begin{cases}
				f(x) \mbox{ if }x \in Z\\
				g^{-1}(x) \mbox{ if } x \in A \setminus Z
				\end{cases}
				\end{align*}
				$h$ is injective, since given $x,y \in A$ such that
				$h(x) = h(y)$, we get on of the three case:
				\begin{itemize}
					\item If $x,y \in Z$ and $f(x) = f(y)$, then $x = y$ since $f$ is injective.
					\item If $x,y \in A \setminus Z$ and $g^{-1}(x) = g^{-1}(y)$ then 
					$x = g(g^{-1}(x)) = g(g^{-1}(y)) = y$.
					\item If $x \in Z$, $y \in A \setminus Z$, then
					$h(x) = f(x) \in f[Z]$ and $h(y) = g^{-1}(y) \in B \setminus f[Z]$, but since $A \setminus Z = g[B \setminus f[Z]]$, we get a contradiction.
				\end{itemize}
				We are left with showing that $h$ is surjective. Let $y \in B$. If $y \in f[Z]$ we are done by the definition of $f[\cdot]$, that there exists $x \in Z$ such that
				$h(x) = f(x) = y$.
				If $y \in B \setminus f[Z]$ and $g(y) \in g[B \setminus f[Z]] = A\setminus Z$, then $h(g(y)) = g^{-1}(g(y)) = y$.
			\end{solution}
			
			%		\item What are the smalles and largest fixed points of H~?
			%		(Comparer avec la preuve du cours)
		\end{enumerate}
		
		\begin{solution}
			
			Assuming that $A$ and $B$ are disjoint.  For all $x\in A\cup B$ define the following sequence :
			\begin{itemize}
				\item $x_0 := x$
				\item For all $n \in \N$, if $x_{n}$ is defined:
				\begin{itemize}
					\item if  $x_n\in g[B]$, then $x_{n+1} := g^{-1}(x_{n})$.
					\item if $x_n\in f[A]$, then $x_{n+1} := f^{-1}(x_{n})$.
				\end{itemize}
			\end{itemize}
			
			Let $A'$(resp.  $B'$) be the set of elements $x\in A$(resp.  $x\in B$) such that the sequence $(x_n)$ terminates in $B$(resp.  $A$). 
			(For all $x \in (A \cup B) \setminus (A' \cup B')$, the sequences $(x_n)$
			terminate in $A$ or are infinite.)
			$A'$ is the least fixed point of $H$.
			The greatest fixed point of $H$ est
			\[
			A' \cup \{ x \in A \mid \text{ the
				infinite sequances $(x_n)$}  \}
			\]
		\end{solution}
		
		% \textbf{Rappel (Continuité au sens de Scott)~:}\\
		% Soit $E$ et $F$ deux ensembles ordonnés. $f: E \to F$ est continue au
		% sens de Scott si pour toute suite croissante $(x_n)_{n \in \N}$ admettant
		% une borne supérieure, $(f(x_n))_{n \in \N}$ en admet aussi une et
		% $\sup_{n \in \N}f(x_n) = f(\sup_{n \in \N}x_n)$.
		
		\qformat{\textbf{Exercise \thequestion \,:}\\}
		\question
		Show that any function from $\R$ to $\R$, is Scott continuous iff it is left continuous and monotonically increasing.
		\begin{solution}
			Let $f$ be Scott-continuous function from $\R$ to $\R$.
			\begin{itemize}
				\item Let $y<z$. Suppouse that $x_0=y$ and $x_n=z$ for all $n \geq  1$.
				We have $\sup_{n \in \N}f(x_n) = f(\sup_{n \in\N}x_n)$, i.e. $\sup\{f(y),f(z)\} = f(z)$, therefore $f(y) \leq f(z)$, and	$f$ is monotonically increasing.
				\item Let $c\in\R$. Let $(x_n)$ be a sequence in $\R$ increasing and converging to $c$.
				We have $c = \sup \{ x_n \}$, therefore $f(c) = \sup \{ f(x_n) \}$ by Scott-continuity.
				Moreover, the sequence $(f(x_n))$ is increasing because $f$ an $(x_n)$ are increasing. 
				Hence $f(x_n) \rightarrow f(c)$, and $f$ is left continuous in $c$.
			\end{itemize}
			
			On the other hand, let $f$ be a monotonically increasing function from $\R$ to $\R$, and left continuous in $c\in\R$.
			Let $(x_n)$ an increasing sequence in  $\R$ s.t. $\sup \{ x_n \} = c$.
			we have $x_n \rightarrow c$, hence $f(x_n) \rightarrow f(c)$ by continuity on the left of $f$ on $c$.
			Moreover, since $f$ in increasing, $(f(x_n))$ is increasing, therefore $\sup \{ f(x_n) \} = f(c)$.
			and $f$ is Scott-continuous in $c$.
		\end{solution}
		\section{Mid-term 2019}	
		
		\question
		
		For all  $n,k \in \N$ such that $k \leq n$, we denote by $\binom{n}{k}$ the number of subsets of $[n]$ with cardinality $k$. By using the following formula
		$$(k+1) \binom{n+1}{k+1} = (n+1) \binom{n}{k}$$
		for al $n,k \in \N$ such that $k \leq n$, show that
		$$\binom{n}{k} = \frac{n!}{k!(n-k)!}$$
		for all $n,k \in \N$ such that $k \leq n$. 
		
		
		\begin{solution}
			For all $k \in \N$, let	 $P_k$ the following property : 
			
			$$\forall n \in \N,\, k \leq n \Rightarrow \binom{n}{k} = \frac{n!}{k!(n-k)!}$$
			
			We show $P_k,\forall k \in \N$ by induction on $k$.
			
			For $k = 0$ : For all $n \in \N$, the empty set is the only subset of $\llbracket 1,n\rrbracket$ with cardinality $0$, hence $\binom{n}{0} = 1$. 
			
			Let $k \in \N$ be such that $P_k$ is verified. For all $n \in \N$ such that $k+1 \leq n$ (where $1 \leq n$), we compute
			\begin{align*}
			\binom{n}{k+1} & =  \frac{n}{k+1} \binom{n-1}{k} \mbox{ form the given formula}\\
			& = \frac{n}{k+1} \cdot \frac{(n-1)!}{k!(n-1-k)!} \mbox{ since }P_k\mbox{ holds by induction }\\
			& = \frac{n!}{(k+1)!(n-(k+1))!}
			\end{align*}
			therefore, $P_{k+1}$ holds.
			
		\end{solution}
		
		\question
		A finite complete ternary tree is a finite tree such that any node has either zero or three children. More formally, a ternary tree is composed either of only a leaf, denoted $ [] $, or of a root having three children $ T_1, T_2, T_3 $, which is denoted by $ [T_1, T_2, T_3] $. A childless node is a leaf, the others are internal nodes.
		
		\begin{enumerate}
			\item Conjecture a formula linking the number of leaves $\ell(T)$ and the number of internal nodes $ i(T) $ in a finite ternary tree $ T $.
			
			\item Define recursively (on the structure of the tree) the functions $ \ell $ and $ i $.
			
			\item Prove the formula conjectured previously by structural induction.
		\end{enumerate}
		
		\begin{solution}
			\begin{enumerate}
				\item The formula is $\ell(T) = 2i(T)+1$. 
				\item 
				\begin{itemize}
					\item $\ell([]) := 1$ and $\ell([T_1,T_2,T_3]) := \ell(T_1) + \ell(T_2) + \ell(T_3)$.
					\item $i([]) := 0$ and $i([T_1,T_2,T_3]) := i(T_1) + i(T_2) + i(T_3) + 1$.
				\end{itemize}
				\item If the tree $ T $ is a leaf, we have $\ell (T) = 1 = 2 \cdot 0 + 1 = 2i (T) + 1 $. Otherwise, $ T = [T_1, T_2, T_3] $ and we have
				\begin{align*}
				\ell(T) & = \ell(T_1) + \ell(T_2) + \ell(T_3) = 2i(T_1) + 1 + 2i(T_2) + 1 + 2i(T_3) + 1\\
				& = 2(i(T_1) + i(T_2) + i(T_3) + 1) +1\\
				& = 2i(T) +1
				\end{align*}
			\end{enumerate}
		\end{solution}
		
		
		\question
		Let $ D \subseteq \Z $. A function $ f: D \to \Z $ is convex (or concave) if for every $ x, y, z \in D $ such that $ x <y <z $, we have $ \frac{f (y ) - f (x)}{y - x} \leq \frac{f (z) - f (y)}{z-y} $ (resp. $ \frac{f (y) - f (x)}{y - x} \geq \frac{f (z) - f (y)}{z-y} $). A function $ f: D \to\Z $ is affine if it is convex and concave.
		
		\begin{enumerate}
			\item What is the cardinality of the set of affine functions from $ \Z $ to $ \Z $, countable or uncountable?
			\item  Are the following two assertions equivalent? (If not, does one imply the other?)
			\begin{enumerate}
				\item $f : \Z \to \Z$ is convex.
				\item For all $n \in \Z$ we have $f(n+1) -f(n) \leq f(n+2)-f(n+1)$.
			\end{enumerate}
			
			
			\item  What is the cardinality of the set of convex functions from $ \Z $ to $ \Z $, countable or uncountable?
			
			\item Let $ f: \Z \to \Z $. Show that there exists an infinite subset $ D \subseteq \Z $ such that $ f|_D: D \to \Z $ (i.e. $ f|_D(n): = f (n) $ for all $ n \in D $) has the following two properties:
			\begin{itemize}
				\item $f|_D$ is increasing or decreasing, and
				\item $f|_D$ is convex or concave.
			\end{itemize}
		\end{enumerate}
		
		
		\begin{solution}
			\begin{enumerate}
				\item An affine function $ f: \Z \to \Z $ is completely characterized by $ f (0) $ and $ f (1) $. There is a countable number of choices for both. Since $ \N \times \N $ is countable, we get the answer.
				
				\item It's equivalent. $ (a) \Rightarrow (b) $ is clear. So suppose $ (b) $. Let $ n \in \Z $. We show that
				
				$$ \forall k \in \N, \, f(n + 1) -f (n) \leq f (n + 1 + k) - f(n + k) $$
				
				by induction on $ k $. The base case $ k = 0 $ is trivial. By the induction hypothesis:
				$$ f (n + 1) -f (n) \leq f (n + 1 + k) - f (n + k) \leq f (n + k + 2) - f (n + k +1) $$ 
				Therefore,
				\begin{align}
				\forall n,p \in \Z,\,n < p \Rightarrow f(n+1)-f(n) \leq f(p+1) -f(p) \label{ineq3.1} 
				\end{align}
				Let $x,y,z \in \Z$ such that $x < y < z$. We get,
				\begin{align*}
				\frac{f(y)- f(x)}{y-x} & = \frac{1}{y-x}\sum_{i = 0}^{y-x-1}f(x+i+1)-f(x+i)\\
				& \leq (f(y) - f(y-1)) \mbox{ by (\ref{ineq3.1})}
				\end{align*}
				in the same way,
				\begin{align*}
				(f(y) - f(y-1)) &\leq \frac{1}{z-y} \sum_{i = 0}^{z-y-1} f(y+i+1) - f(y + i) \mbox{ by (\ref{ineq3.1})}\\
				& = \frac{f(z) - f(y)}{z-y}
				\end{align*}
				Combining the two inequalities above by transitivity proves convexity. 
				
				
				\item For any infinite bit sequence $ \alpha \in \{0,1 \}^\omega $, we define the function $ f_\alpha: \Z \to \Z $ as follows:
				\begin{itemize}
					\item $f(n) := 0$ for all $n \in \Z \cap (-\infty,0]$,
					\item $f(1) := 1$,
					\item $f(n+2) := 2f(n+1) - f(n) + \alpha_n$ for all $n \in \N$.
				\end{itemize}
				We check that		
				$f(n+2)-f(n+1) = f(n+1) - f(n) + \alpha_n \geq f(n+1) - f(n)$. By the previous question, $ f $ is convex. Let's be two separate sequences $ \alpha, \beta \in \{0,1 \}^\omega $. Let $ n $ be the length of their largest common prefix. Then $ f_\alpha(n + 1) \neq f_\beta(n + 1) $. The set is therefore uncountable.
				
				\item We first extract a monotonic function of $ f $ by invoking the infinite Ramsey theorem. We define $ \gamma: \mathcal{P}_2 (\Z) \to \{c, d\} $ as follows: for all $ n, p \in \Z $ such that $ n <p $, we put $ \gamma (\{n, p \}): = c$ if $ f(n) <f (p) $, if not $ \gamma (\{n, p \}): = d $. According to Ramsey's theorem, there exists an infinite subset $ D_0 \subseteq \Z $ such that for all $ | \gamma (D_0) | =  1$.
				
				We will invoke Ramsey's theorem a second time. We define $ \delta: \mathcal{P}_3 (D_0) \to \{c, d \} $ as follows: for all $ x, y, z \in D_0 $ such that $ x <y < z $, we put $ \delta (\{x, y, z \}): = c $ if $ \frac{f (y) - f (x)}{y - x} \leq \frac{f ( z) - f (y)}{z-y} $, otherwise $ \delta(\{x, y, z \}): = d $. Again, there is an infinite subset $ D\subseteq D_0 $ such that $ | \delta [D]| = $ 1. As $ D \subseteq D_0 $, we have $ | \gamma [D] | = $ 1, which allows to conclude.
			\end{enumerate}
		\end{solution}
		
		
		
		\question
		\,
		\begin{enumerate}
			\item Let $ (E, \leq) $ be a partially ordered set. Let $ \mathcal{C} $ be the set of well-founded chains of $ (E, \leq) $. We define a binary relation $ \mathrel{R} $ on $ \mathcal{C} $ as follows: for all $ C_1, C_2 \in \mathcal{C} $ we put $ C_1 \mathrel{R} C_2 $ if $ C_1 \subseteq C_2 $ and for all $ x \in C_1 $ and $ y \in C_2 \setminus C_1 $ we have $ x \leq y $.
			\begin{enumerate}
				\item Which type of relation is $\mathrel{R}$ ?
				
				\item Show that all the chains $\{C_i\}_{i \in I}$ in $(\mathcal{C},\mathrel{R})$ have a least upper bound in $(\mathcal{C},R)$.
				
				\item Deduce that there is a chain of $(E,\leq)$ which is a maximal element of $(\mathcal{C},\mathrel{R})$.
			\end{enumerate}
			
			\begin{solution}
				\begin{enumerate}
					\item $R$ est un ordre (partiel).
					
					\item Let $ C: = \cup_ {i \in I} C_i$. $ C $ is well founded because a descendant chain of $C$ is included in a $C_i$. Let $ i \in I $. On the one hand, $ C_i \subseteq C $. On the other hand, let $ x \in C_i $ and $ y \in C \setminus C_i $. Then there exists $ j \in I \setminus \{i \} $ such that $ y \in C_j \setminus C_i $, so $ \neg (C_j \subseteq C_i) $, or $ \neg (C_j R C_i) $. Now $ \{C_i \} _ {i \in I} $ is a chain, so $ C_i R C_j $, hence $ x \leq y $. So $ C_iRC $, so $ C $ is an upper bound of $ \{C_i \}_{i \in I} $.
					
					Let $ D \in \mathcal {C} $ be an upper bound of $ \{C_i \}_{i \in I} $. Then $ C_i \subseteq D $ for all $ i \in I $, so $ C \subseteq D $. Let $ x \in C $ and $ y \in D \setminus C $. Then there is $ i \in I $ such that $ x \in C_i $. Now $ y \in D \setminus C_i $, so $ x \leq y $. Hence, $ C R D $, which shows that $ C $ is the desired upper bound.
					
					
					
					\item According to Zorn's Lemma(Suppose a partially ordered set P has the property that every chain in P has an upper bound in P. Then the set P contains at least one maximal element), the ordered set $ (\mathcal{C}, R) $ thus has a maximal element $ C $, which is a chain of $ (E, \leq) $.
				\end{enumerate}
				
			\end{solution}
			
			
			
			\item Let $(E, \leq)$ a lattice for which any well-founded chain has an upper bound.
			
			\begin{enumerate}
				\item Show that $(E, \leq)$ has a least element $\bot$.
				
				\item Let $ A $ be a subset of $ E $. Let $ B $ be the set of all elements smaller that all elements in $A $ . Show that if $ B $ has a maximum element $ b $, then $ b $ is the greatest element of $ B $.
				
				
				\item Show that $B$ has a maximal element $b$. (you may use question 7.1.c)
				
				\item Show that $(E, \leq)$ is a complete lattice.
			\end{enumerate}
			
			
			\begin{solution}
				\begin{enumerate}
					\item The empty set has an upper bound, which is therefore also the least element.
					
					\item Let $b' \in B$. Let $d := \sup \{b,b'\}$. By definition of $B$, we have $b \leq a$ and $b' \leq a$ for all $a \in A$, hence $d \leq a$ by the definition of suprimum. Observe that $d \in B$. Since $b \leq d$, we get that $b = d$ by the maximallty of $b$, and therefore $b' \leq b$. 
					
					
					\item $(B, \leq_B)$ is a non empty po, since $\bot \in B$. by question 7.1.c, let $C$ be a maximal well founded chain in $(B, \leq_B)$. By the hipothesis of the question, $C$ has an upper bound $b$. Let $b' \in B$ such that $b < b'$.  Then the set $C \cup \{b'\} \neq C$ is a well founded chain in $(B, \leq_B)$ such that $CR(C \cup \{b'\})$, which contradicts our assumption. Therefore $b$ is a maximal element of $B$.
					
					\item $b$ is the infimum of $A$. therefore all sets in $E$ have an infimum and therefore $E$ is a complete lattice.
				\end{enumerate}
				
			\end{solution}
		\end{enumerate}
		
		
	\end{questions}
	
\end{document}
