\documentclass[a4paper,11pt]{exam}
%\printanswers % pour imprimer les réponses (corrigé)
% \noprintanswers % Pour ne pas imprimer les réponses (nonc)
% \addpoints % Pour compter les points
% \noaddpoints % pour ne pas compter les points
\qformat{\textbf{Exercice \thequestion \,:}\\} % Questions style
\usepackage{color} % Define new colors

\usepackage[utf8x]{inputenc}
\usepackage[T1]{fontenc}
\usepackage{exercise}
\usepackage{mathrsfs,amsmath,amssymb,latexsym,amsfonts,mathtools,stmaryrd}
\usepackage{enumitem}
\setitemize{label=\textbullet}
% Bigger binomial in math mode
\usepackage{nccmath}
\renewcommand{\binom}{\mbinom}

\usepackage[francais]{babel}

\usepackage{tikz}
\usetikzlibrary{trees}

\shadedsolutions % définit le style des réponses
% \framedsolutions % définit le style des réponses 
\definecolor{SolutionColor}{rgb}{0.8,0.9,1} % bleu ciel
\renewcommand{\solutiontitle}{
\noindent\textbf{Solution:}\par\noindent} % Définit le titre des réponses
\newcommand\eqdef[0]{\stackrel{\text{def}}{=}}

\pagestyle{headandfoot}
\headrule
\header{L3 - 2019/2020 - TD7}{Mercredi 06 November}{Mathématiques discrètes}
\footer{S. Le Roux, I. Khmelnitsky}{\thepage}{E.N.S. Cachan}
\footrule

\renewcommand{\questionshook}{%
  \setlength{\labelwidth}{0pt}%
  \setlength{\itemsep}{0.9\baselineskip}
}

\definecolor{gris}{gray}{0.95}
\newcommand{\Z}{\mathbb{Z}}
\newcommand{\N}{\mathbb{N}}
\newcommand{\Q}{\mathbb{Q}}
\newcommand{\R}{\mathbb{R}}
\newcommand{\C}{\mathbb{C}}
\newcommand{\F}{\mathbb{F}}
\newcommand{\A}{\mathcal{A}}
\renewcommand{\S}{\mathcal{S}}
\newcommand{\RR}{\mathbin{\mathcal{R}}}
\renewcommand{\P}{\mathrm{P}}
\DeclareMathOperator{\pgcd}{pgcd}
\DeclareMathOperator{\Fix}{Fix}
\DeclareMathOperator{\fix}{Fix}

\begin{document}
	Let $\Sigma$ a finite alphabet.
\begin{questions}
	

  \qformat{\textbf{Exercice \thequestion~(\thequestiontitle):}\\}
  \titledquestion{Syntactic monoid}
  Let $L \subset \Sigma^*$ be a language. This defines the equivalence relation on $\Sigma^*$~:
 $
    w \sim_L w' \Leftrightarrow \forall u, v\in \Sigma^*, \; uwv\in L
    \Leftrightarrow uw'v\in L
  $.
  Justify that $\sim_L $ is a congruence on $\Sigma^*$
  ($x\sim y$ iff $uxv\sim uyv \,\,\forall v,u\in M$).
  We define the Syntactic monoid $M_L$ as the quotient 
  $\Sigma^*|_{\sim_L}$.

  \begin{solution}
    Let $w,w'$ s.t $w \sim_L w'$.
    Let $s,t \in \Sigma^*$.
    $\forall u,v \in \Sigma^*, u(swt)v=(us)w(tv)$ hence
  ${u(swt)v \in L} \Rightarrow (us)w'(tv) \in L$.
    Since $(us)w'(tv)=u(sw't)v$, we get that $u(sw't)v \in L$.
    By symmetry, $u(swt)v \in L \iff u(sw't)v \in L$, hence $swt \sim_L sw't$.
  \end{solution}

%  \titledquestion{Language recognized by a monoid}
%  Let $L \subset \Sigma^*$ a language.
%  Let $M$ be a monoid. We say that a language $L$ is recognizable by $M$ if there exists a monoid morphism
%  $\varphi$ from $\Sigma^*$ to $M$ and a subset $X$ of $M$ such that  $L = \varphi^{-1}(X)$.
%  \begin{enumerate}
%    \item Show that a language recognized by a finite monoid is regular.
%
%      \begin{solution}
%        Let $L$ be recognized by $M$ using the monoid morphism $\varphi$ of $\Sigma^*$ to $M$ and $X$ a subset of $M$ such that $L = \varphi^{-1}(X)$.
%        Then $L$ is the language recognized by an automaton where the states are element of $M$, the initial state $1$, the final states are elements in $X$, and the transitions are ~:
%        \[
%          \forall m \in M, \forall a \in \Sigma, m \xrightarrow{a} m\varphi(a)
%        \]
%      \end{solution}
%
%    \item Show that a language  $L$ is recognized by its syntactic monoid.
%
%      \begin{solution}
%        let $\varphi$ the surjection of $\Sigma^*$ to
%        $\Sigma^*|_{\sim_L}$.
%        Then $L=\varphi^{-1}(\varphi(L))$.
%      \end{solution}
%
%    \item Show that a language  $L$ is recognized by a monoid $N$ iff $M_L$ is isomorphic to a quotient of a sub monoid of $N$.
%
%      \begin{solution}
%%      	et L  M using a monoid morphism φ of Σ * in M and X a
%%      	part of M such that L = φ -1 (X). So L is the language recognized by the automaton
%%      	the states are the elements of M, the initial state is 1, the final states are the elements
%%      	of X, and the transitions are
%%      	
%      	
%        Let $L$ be recognized by $N$ using a monoid morphism $\varphi$
%        from $\Sigma^*$ to $N$ and $X$ a subset of $N$ s.t. $L = \varphi^{-1}(X)$.
%        Let $w,w' \in \Sigma^*$.
%        If $\varphi(w)=\varphi(w')$, then
%        $\forall u,v \in \Sigma^*, uwv \in L \iff
%        \varphi(uwv) \in X \iff \varphi(u)\varphi(w)\varphi(v) \in X \iff \varphi(u)\varphi(w')\varphi(v) \in X \iff \varphi(uw'v) \in X \iff
%        uw'v \in L$
%        and hence $uwv \in L \iff uw'v \in L$.
%        We have shown that~: $\varphi(w) = \varphi(w') \Rightarrow w \sim_L w'$. Hence $\varphi$ is a well defined 1-1 morphism from $M_L$ to $N$, and hence isomorphic to some sub monoid.
%      \end{solution}
%      
%    \item Deduce the characterization of regular languages relating to their syntactic monoid.
%
%      \begin{solution}
%      	If a language is  regular then it has an automate and hence there is a finite monoid($\Sigma^* $ with equivalence according to the states)  which recognize it.
%      	
%        A language is regular if and only if its syntactic monoid is finite.
%      \end{solution}
%  \end{enumerate}

 \titledquestion {Star-free languages}
%  Soit $\Sigma$ un alphabet fini.
%  La famille des langages sans étoile est la plus petite famille contenant le
%  langage vide, les singletons et stable par union, passage au complémentaire
%  et concaténation.
Let $\Sigma$ be a finite alphabet.
The Star-free  family of languages is the smallest family containing empty language, singletons and stable by union, passage to complement and concatenation.
  \begin{enumerate}
  	 \item Show that $\Sigma^*$ is star-free.
  	
  	\begin{solution}
  		$\Sigma^*$ is the complementary of the empty language.
  	\end{solution}
    \item Demonstrate that the intersection of two star-free languages is star-free.
      \begin{solution}
        \[
          L_1 \cap L_2 =(L_1^c\cup L_2^c)^c =  \Sigma^* \setminus
          (\Sigma^* \setminus L \cup \Sigma^* \setminus L')
        \]
      \end{solution}

    \item Let $a,b \in\Sigma$ where $a\neq b$. Show that $(ab)^*$ is star-free.

      \begin{solution}
        \[
          (ab)^*=(a\Sigma^* \cap \Sigma^*b) \setminus 
          (\Sigma^* a^2 \Sigma^* \cup \Sigma^* b^2 \Sigma^*)
        \]
      \end{solution}
  \end{enumerate}

  
  We call a finite monoid aperiodic if the only group it contains is the trivial group $\{1\}$.
  \begin{enumerate}[resume]
    \item Let $M$ be a finite monoid.
      Show that the following are equivalent:
      \begin{enumerate}
        \item The monoid $M$ is aperiodic.
        \item For all $m$ in $M$, there exists a nonzero natural number $n$  such that $m^{n+1} = m^n$,
        \item There exists a non zero natural number $n$ such that for all $m$ in $M$, $m^{n+1} = m^n$.
      \end{enumerate}

      \begin{solution}
        $(a) \Rightarrow (b)$: Suppose there is $m \in M$ s.t.for all $n>0$, $m^{n+1} \neq m^n$.
        By th pegion hole principle, there exists an integer $0<k<l$ s.t. $m^k = m^l$. We take $k$ the smallet possible and set $p=l-k$. So $m,m^2, \dots, m^{l-1}$ are different elements and $\{m^k, \dots, m^{l-1}\}$ is a group of order $p$ (of natural elements $m^r$, $r$ is a multiple between $p$ and $k$ and $l-1=k+p-1$).
        By the hypothesis we know that $p \geq 2$.

        $(b) \Rightarrow (c)$: take the maximal from all the elements in $M$.

        $(c) \Rightarrow (a)$: if there exists $n>0$ s.t.
        $\forall m \in M, m^{n+1} = m^n$, the only inversible element in $M$ is $1$ since $m^{n+1}m^{-n} = m^nm^{-n} \Rightarrow m = 1$,  hence $M$ is aperiodic.
      \end{solution}

    \item Let $L$ be a regular language and let $M_L$ be its syntactic monoid.
      By the definition of a syntactic monoid, we deduce by the previous question that $M_L$ is aperiodic if and only if, for all words $u$, there exists a non-zero natural number $n$ s.t. for all words $v,w$, $vu^nw \in L \Leftrightarrow vu^{n+1}w \in L$.
      In this case we denote it by $i(L)$ the smallest natural non-zero number $n$ such that for all words $u,v,w$, $vu^nw \in L \Leftrightarrow vu^{n+1}w \in L$. 
      \begin{enumerate}
        \item Show the following(for languages $L$ with finite $i(L)$):
          \begin{enumerate}
            \item $i(\{a\}) = 2$,
            \item $i(L_1\cup L_2) \leq \max(i(L_1), i(L_2))$,
            \item $i(L_1L_2) \leq i(L_1)+ i(L_2)+1$,
            \item $i(\Sigma^* \setminus L) = i(L)$.
          \end{enumerate}
      \begin{solution}
      	Denote by $n_1=i(L_1), n_2=i(L_2)$ and $n=n_1+n_2+1+1$.
      	For any $x,u,v\in \Sigma^* $ we show that $ux^nv\in L_1L_2$ implies that  $ux^{n+1}v\in L_1L_2$. 
      	We suppose that $x\neq e$. If $ux^nv\in L_1L_2$ then $ux^nv = x_1x_2$ where $x_1\in L_1, x_2\in L_2$. 
      	There are 2 cases, either $x_1 = ux^{n_1}r$ with $rx_2=x^{n-n_1}v$ or $x_2=sx^{n_2}v$ with $x_1s=ux^{n-n1}$. Consider the first case, since ---****--- $ux^{n_1}v\in L_1$ we have $ux^{n_1+1}v\in L_1$ and hence $ux^{n+1}v\in L$ 
      \end{solution}
      
        \item Deduce that a syntactic monoid of a star-free language is aperiodic.
      \end{enumerate}
   

  \end{enumerate}

 \question
If $M$ is a monoid and $K, L$ two subsets of $M$, we denote by  $L^{-1}K = \{ x\in M \mid \exists y \in L, yx \in K \}$ and  $KL^{-1} = \{ x\in M \mid \exists y \in L, xy \in K \}$.
\begin{enumerate}
	\item Let $L$ a sub monoid of $\Sigma^*$. Show that $L$ is a free monoid iff $L^{-1}L \cap LL^{-1}=L$.
	
	\begin{solution}
		Suppose that $L$ is free on the set $B$. 
		Let $m \in L^{-1}L \cap LL^{-1}$.
		There exist $p,q$ in $L$ s.t. $mq,pm \in L$.
		Moreover since $(pm)q=p(mq)\in L$, we get that $m\in L$

		\smallskip
		
		Conversely, suppose that $L^{-1}L \cap LL^{-1}=L$. Let $B$ be the minimal generating parts of $L$ (The elements of $L$ which are not a product of two distinct elements, which are not $1$).
		Let $u_1 \dots u_m = v_1 \dots v_n$ with $u_i$ and $v_j$ in $B$.
		Assume for example that in $\Sigma^*$, $u_m=wv_n$, that
		$u_1 \dots u_{m-1} w = v_1 \dots v_{n-1}$, then
		$w \in L^{-1}L \cap LL^{-1}=L$.
		By the minimialty of elements in $B$ of $L$, $w=1$.
		we conclude by recurrence.
	\end{solution}
	
	\item Let $L$ be a sub-monoid of $\Sigma^*$. We define by recursion~:
	\begin{itemize}
		\item $M_0 = L$
		\item $M_{n+1} = \langle {M_n}^{-1}{M_n} \cap {M_n}{M_n}^{-1} \rangle$
	\end{itemize}
	Demonstrate that this is a well defined increasing sequence and that $\cup_N M_n$ is the smallest free sub-monoid containing $L$.
	
	\begin{solution}
		We note that for any $M$,
		$M \subset M^{-1}M \cap MM^{-1}$ ($\forall u \in M, 1u \in M$ and
		$u1 \in M$) therefore $(M_n)_n$ is well defined increasing sequence of monoids.
		Hence $M = \cup_n M_n$ is a monoid.
		
		We show that $M^{-1}M \cap MM^{-1} \subset M$~:
		let $u \in \Sigma^*$ s.t. there exists $v$ and $w$ in $M$ s.t. $vu \in M$ and $uv \in M$.
		$M = \cup_n M_n$, hence there exists an integer $l$ and $m$ s.t. $v \in M_l$ and $w \in M_m$.
		For $n=\max(l,m)$, $v$ and $w$ are in $M_n$, hence
		$u \in M_n^{-1}M_n \cap M_n M_n^{-1} \subset M_{n+1} \subset M$.
		Therefore $M$ is free.
		
		Finally, if $N \subset P$ is an inclusion of sub-monoids, with $P$ free, we have $N^{-1}N \cap NN^{-1} \subset P^{-1}P \cap PP^{-1} = P$, so $\langle N^{-1}N \cap NN^{-1} \rangle \subset \langle P \rangle = P$,hence if $P$ contains $L$, it contains all $M_n$ and therefore
		$M$~: $M$ is the smallest free sub monoid containing $L$.
	\end{solution}
\end{enumerate}


 \begin{EnvFullwidth}
	\colorbox{gris}{
		\begin{minipage}[c]{\textwidth}
			Groupes
		\end{minipage}
	}
\end{EnvFullwidth}



%
%\question
%%We denote by $\varphi$ Euler's function.
%
%Let $n$ a natural number $>1$. 
%%Denote by $d(n)$ the number of dividers of $n$.
%
%\begin{enumerate}
%	\item Let $m$ a natural number between $1$ and $n$.
%	Let $H_m$ be the set  of all the elements of $\Z/n\Z$ whose order is a divisor of $ m $, that is, the set of elements $ x $ of $\Z/n\Z$
%	such that $\overline{m} x = \underbrace{x + x + \cdots + x}_{m} = 0$.
%	Show~:
%	\begin{enumerate}
%		\item $H_m$ is a sub group  of $\Z/n\Z$.
%		
%		\begin{solution}
%			Let $H_m = \{ x \in \Z/n\Z \mid
%			\overline{m} x = \underbrace{x + x + \cdots + x}_{m} = 0 \}$
%			
%			Show that $H_m$ is a sub group of $\Z/n\Z$~:
%			\begin{itemize}
%				\item $\forall p\in\N, \overline{p} \cdot 0 = 0$, so
%				$0 \in H_m$.
%				\item Let $x,y \in H_m$.
%				
%				\begin{align*}
%				\overline{m}(x-y) &= \underbrace{x-y + x-y + \cdots + x-y}_m\\
%				&= \underbrace{x+x+\cdots+x}_m - \underbrace{y+y+\cdots+y}_m\\
%				&= 0
%				\end{align*}
%				Hence $x-y \in H_m$.
%			\end{itemize}
%		\end{solution}
%		
%		\item $H_m$ is a cyclic sub-group of $\Z/n\Z$ of cardinality
%		$\textrm{gcd}(m,n)$.
%		
%		\begin{solution}
%			Let $a \in \Z$ relevant $x \in \Z/n\Z$.
%			
%			Then $[x] \in H_m \iff mx \in n\Z \iff x \in (n/\textrm{gcd}(m,n))\Z$
%			(By Gauss theorem).
%			Hence $H_m$ is a cyclic subgroup generated by the class of $n/\mathrm{gcd}(m,n)$ in $\Z/n\Z$.
%			
%			In addition, for any $d \in \N$, $dn/\gcd(m,n) \in n\Z \iff \gcd(m,n) | d$, so the $n/\gcd(m,n)$ class in $\Z/n\Z$ is exactly $\gcd(m,n)$. And so $H_m$ is of order $\gcd(m,n)$.
%			
%			De plus, pour tout $d \in \N$,
%			$dn/\pgcd(m,n) \in n\Z \iff \pgcd(m,n) | d$, donc la classe de
%			$n/\pgcd(m,n)$ dans $\Z/n\Z$ est exactement d'ordre $\pgcd(m,n)$.
%			Et donc $H_m$ est d'ordre $\pgcd(m,n)$.
%		\end{solution}
%		
%		\item Show that the set of all subgroups of $\Z/n\Z$ is exactly all the subgroups $H_d$ for $d \in \N$, $d$
%		is a divisor of of $n$.
%		
%		\begin{solution}
%			For $d$ a  divisor of $n$, $H_d$ is a sub group  of order $d=\gcd(m,n)$.
%			Let $H$ be a sub group of order $d$.
%			Then $H \subset H_d$ by the theorem of Lagrange.
%			Now, $d = |H| = |H_d|$ hence $H=H_d$, hence the uniqueness.
%		\end{solution}
%	\end{enumerate}
%%	\item We consider the following map~:
%%	
%%	\[
%%	\begin{array}{ccccc}
%%	\psi & : & (\Z/n\Z)^* \times \Z/n\Z & \rightarrow & \Z/n\Z \\
%%	&& (\overline{m}, x ) & \to & \overline{m} x 
%%	\end{array}
%%	\]
%%	
%%	\begin{enumerate}
%%		\item Justify that the group $(\Z/n\Z)^*$ acts on $\Z/n\Z$.
%%		
%%		\begin{solution}
%%			$\forall x\in\Z/n\Z, \psi(\overline{1})(x)=\overline{1}x=x$, hence
%%			$\psi(\overline{1})$ is the identity of $\Z/n\Z$.
%%			
%%			$\forall \overline{m_1},\overline{m_2} \in (\Z/n\Z)^*,
%%			\overline{m_1} \cdot \overline{m_2}=\overline{m_1m_2}$
%%			hence $\psi(\overline{m_1m_2})=
%%			\psi(\overline{m_1})\circ\psi(\overline{m_2})$.
%%		\end{solution}
%%		
%%		\item Show the equality~:
%%		
%%		\[
%%		\sum_{\begin{array}{c} m \in \{1,\dots.,n\} \\
%%			\textrm{gcd}(m,n)=1 \end{array}} \textrm{gcd}(m-1,n) =
%%		\varphi(n) d(n)
%%		\]
%%		
%%		\begin{solution}
%%			We show that there are $d(n)$ orbits.
%%			Let $x, y\in\Z/n\Z$.
%%			$x$ and $y$ are in the same orbit under the action of $(\Z/n\Z)^*$ if and only if $\langle x \rangle = \langle y \rangle$
%%			($\exists m\in\N, y=\overline{m}x \Rightarrow y \in \langle x \rangle$).
%%			The $x$ orbit is the set of generators in the $\langle x \rangle$ subgroup.
%%			There are therefore as many orbits as there are cyclic subgroups of $\Z/n\Z$, or $d(n)$ (question 1(c)).\\\\
%%			
%%			
%%	
%%			On démontre qu'il y a $d(n)$ orbites.
%%			Soit $x,y\in\Z/n\Z$.
%%			$x$ et $y$ sont dans une même orbite sous l'action de $(\Z/n\Z)^*$
%%			si et seulement si $\langle x \rangle = \langle y \rangle$
%%			($\exists m\in\N,
%%			y=\overline{m}x \Rightarrow y \in \langle x \rangle$).
%%			L'orbite de $x$ est l'ensemble des générateurs du sous-groupe
%%			$\langle x \rangle$.
%%			Il y a donc autant d'orbites que de sous-groupes cycliques de
%%			$\Z/n\Z$, soit $d(n)$ (question 1(c)).
%%		\end{solution}
%%	\end{enumerate}
%\end{enumerate}




\question
Let $p$ be a prime number.
We denote by $(1,\dots,p)$ the $p$-cycle (the permutation of the symmetrical group that sends $1$ on $2$, $2$ on $3$, \dots, $p-1$ on $p$ and $p$ on $1$).
We denote by $G$ the group generated by the $p$-cycle $(1,\dots,p)$. Assuming that $\Sigma$ a finite alphabet. 
\begin{enumerate}
	\item Show that $G$ is a group of order $p$.
	
	\begin{solution}
		The permutation $\tau=(1,\dots,p)$ is an element of order $p$ so it generates an order group $p$.
		If we want to show that the order of $\tau$ is exactly $p$, we can notice that $\tau(i)$ is congruent to $i+1$ modulo $p$, so $\tau^j(i)$ is congruent to $i+j$ modulo $p$. Thus, $\tau^j=Id$ if and only if $\forall i, \tau^j(i)=i$, so if and only if $j$ is a multiple of $p$.
	\end{solution}
	
	\item What are the orders of the elements in $G$~ ? 
	For each $d$, we will specify how many $G$ elements are of order $d$.
	
	\begin{solution}
		Since $p$ is a prime number, by Lagrange theorem, the elements of $G$ are of order $1$ or $p$.
		Only the neutral element is if order $1$, hence there are $p-1$ elements of order $p$.
	\end{solution}
	
	\item We take $ G $ to act on $ \Sigma^p $, the set of words of length $p $ written with the letters of $ \Sigma $ as follows:
	\[
	\begin{array}{lll}
	G \times \Sigma^p & \rightarrow & \Sigma^p \\
	(\tau, a_1a_2\dots a_p)& \to & \tau \cdot (a_1a_2\dots a_p) =
	a_{\tau^{-1}(1)} a_{\tau^{-1}(2)}\dots a_{\tau^{-1}(p)}
	\end{array}
	\]
	\begin{enumerate}
		\item Demonstrate that this is a group operation.
		
		\begin{solution}
			Let $\sigma$ and $\rho$ be two permutations.
			Let $a_1 a_2 \dots a_p \in \Sigma^p$. 
			Then~:
			\pagebreak
			\begin{align*}
			\sigma \cdot (\tau \cdot (a_1 a_2 \dots a_p))
			&= \sigma \cdot (a_{\tau^{-1}(1)}a_{\tau^{-1}(2)}
			\dots a_{\tau^{-1}(p)}) \\
			&= \sigma \cdot (b_1 b_2 \dots b_p)
			\text{ with } b_i=a_{\tau^{-1}(i)}, \forall i \\
			&= b_{\sigma^{-1}(1)} b_{\sigma^{-1}(2)}
			\dots b_{\sigma^{-1}(p)} \\
			&=  a_{\tau^{-1}(\sigma^{-1}(1))} a_{\tau^{-1}(\sigma^{-1}(2))}
			\dots a_{\tau^{-1}(\sigma^{-1}(p))} \\
			&= a_{(\sigma\tau)^{-1}(1)} a_{(\sigma\tau)^{-1}(2)}
			\dots a_{(\sigma\tau)^{-1}(p)}
			\text{ since } (\sigma\tau)^{-1} = \tau^{-1}\sigma^{-1}
			\end{align*}
		\end{solution}
		
		\item Determine the fixer of $(1,\dots,p)$. Deduce the fixer of $ (1, \dots, p)^i $, for any integer $ i $ co prime with $ p $($\Fix(g)=\{x\in X | gx=x\}$).
		
		\begin{solution}
			Denote $\Fix((1,\cdots,p))$ the fixer  of $(1,\cdots,p)$.
			$a_1 a_2 \dots a_p \in \Fix((1,\cdots,p))$  if
			$(1,\cdots,p) \cdot a_1 a_2 \dots a_p = a_1 a_2 \dots a_p$.
			However $(1,\cdots,p) \cdot a_1 a_2 \dots a_p = a_2 a_3 \dots a_p a_1$.
			Hence $a_1 a_2 \dots a_p \in \Fix((1,\cdots,p))$ iff $a_1=a_2$, $a_2=a_3$, \dots, $a_p=a_1$.
			Hence $\Fix((1,\cdots,p)) = \{ a^p \mid a \in \Sigma \}$.
			
			A word in the fixer of $(1,\cdots,p)$ is in the fixer of $(1,\cdots,p)^i$, for all integers $i$.
			If $i$ is co prime with $p$, there exist  integers $a$ an $b$ s.t. $ap+bi=1$ hence $(1,\cdots,p)=((1,\cdots,p)^p)^a((1,\cdots,p)^i)^b=((1,\cdots,p)^i)^b$, hence the word which is in the fixer of $(1,\cdots,p)^i$ is in the fixer of $(1,\cdots,p)$.
			Therefore for all $i$ coprime with $p$, $\Fix((1,\cdots,p))=\Fix((1,\cdots,p))^i$.
		\end{solution}
		
		\item Show that the number of orbits, $r$, of this operation are~:
	$
		r = \frac{1}{p} (|\Sigma|^p + (p-1) |\Sigma|)
	$
		
		\begin{solution}
			By Burnside~:
			\[
			r = \frac{1}{|G|}\sum_{g \in G}|\Fix(g)|
			\]
			The identity fixes every element of  $\Sigma^p$, the other elements of $G$ are $(1,\cdots,p)^i$ for $i \in \{1,\cdots,p-1 \}$ and their fixer $\{ a^p \mid a \in \Sigma \}$, is of cardinality $|\Sigma|$.
			And $G$ is of order $p$, so the formula of Burnside gives us~:
			\[
			r = \frac{1}{p}(|\Sigma|^p+(p-1)|\Sigma|)
			\]
		\end{solution}
		
		\item Retrieve Fermat's little theorem ($a^{p} \equiv_p a$).
		
		\begin{solution}
			Since the number of orbits is a natural number, we deduce that  $p$ divides the integer $|\Sigma|^p+(p-1)|\Sigma| =
			(|\Sigma|^{p-1}+(p-1))|\Sigma|$.
			When the alphabet of $\Sigma$ is of cardinality co prime to $p$, $p$ is inverse modulo $p$ and $p-1 \equiv -1 \mod p$, hence $|\Sigma|^{p-1} \equiv 1 \mod p$.
			For all $n\in\N$ s.t. $n \wedge p = 1$ an alphabet of cardinalty $n$, retrives Fermat's little theorem.		\end{solution}
	\end{enumerate}
\end{enumerate}
\begin{EnvFullwidth}
	\colorbox{gris}{
		\begin{minipage}[c]{\textwidth}
			Probability
		\end{minipage}
	}
\end{EnvFullwidth}
\question
Show that for all sequences $(A_n)_{n \in \N}$ of events we have
$
\P(\cup_{n \in \N}A_n) \leq \sum_{n \in \N}\P(A_n)
$
where the right sum can diverge.

\begin{solution}
	For all $n \in \N$, let $B_n := A_n \setminus \cup_{k = 0}^{n-1}A_k$.
	We show by recurrences on $n$ that $\cup_{k = 0}^nB_k = \cup_{k = 0}^nA_k$.
	We deduce that $B_n \cap \cup_{k = 0}^{n-1}B_k = \emptyset$.
	
	Hence $\P(\cup_{n \in \N}A_n) = \P(\cup_{n \in \N}B_n)
	= \sum_{n \in \N}\P(B_n) \leq \sum_{n \in \N}\P(A_n)$,
	because $B_n \subseteq A_n$.
\end{solution} 

\question
Let $(\Omega,\mathcal{T},\P)$ be a probability space.
\begin{enumerate}
	\item Show that for all sequences $(A_n)_{n \in \N}$ of growing events by inclusion, the sequence of $\P(A_n)$ converges and
	$
	\lim_{n \to \infty} \P(A_n) = \P(\cup_{n \in \N}A_n)
	$
	
	
	\begin{solution}
		$\cup_{n \in \N}A_n = A_0 \sqcup \sqcup_{n \in \N}A_{n+1}\setminus A_n$,
		hence $\P(\cup_{n \in \N}A_n)
		= \P(A_0) + \sum_{n \in \N}\P(A_{n+1} \setminus A_n)$ therefore
		$\sum_{i = n}\P(A_{i+1}\setminus A_i) \to_{n \to \infty}0$.
		
		We get that $\P(\cup_{n \in \N}A_n)
		= \P(A_k) + \sum_{n = k}^{+\infty}\P(A_{n+1} \setminus A_n)$, therefore
		$\P(A_n) \to_{} \P(\cup_{n \in \N}A_n)$
	\end{solution}
	
	\item Show for all decreasing sequences  (by inclusion) $(A_n)_{n \in \N}$ of events, the sequence $\P(A_n)$ converges and
	$
	\lim_{n \to \infty} \P(A_n) = \P(\cap_{n \in \N}A_n)
	$
	
	\begin{solution}
		If the sequence is decreasing, then $A_0 \setminus A_n$ is increasing.
		$\P(A_n) + \P(A_0 \setminus A_n) = \P(A_0)
		= \P(\cap_{n}A_n) + \P(\cup_{n}A_0\setminus A_n)$.
		Since $\P(A_0 \setminus A_n) \to \P(\cup_{n}A_0\setminus A_n)$ by the previous results we get $\P(A_n) \to \P(\cap_{n}A_n)$.
	\end{solution} 
\end{enumerate}

\question
Let $\{B_1,\dots,B_n\}$ be a partition of $\Omega$ such that for all $i$,
$\P(B_i) > 0$.
Show that for all events $A$ we have
$
\P(A) = \sum_{i = 1}^n \P(A \mid B_i) \P(B_i)
$. 
Keeping $B_i$'s disjoint, what condition condition can we add for this to remain true?
\begin{solution}
	$\P(A)
	= \P(A \cap \sqcup_iB_i)
	= \P(\sqcup_i(A \cap B_i))
	= \sum_{i} \P(A \cap B_i)
	= \sum_{i} \P(A \mid B_i)\P(B_i)$.
\end{solution} 

\question
An urn contains $ b $ black balls, $ w $ white balls and $ r $ red balls.
We pick two balls, what is the probability of the event ``the second ball drawn is
black'' ~?

\begin{solution}
	Let $S = b + w + r$, let $SB$ the event ``the second ball is black'', and $B$, $W$, $R$ for the first ball.
	\begin{align*}
	\P(SB)
	&= \P(SB \mid B)\cdot\P(B) + \P(SB \mid W)\cdot\P(W)
	+ \P(SB \mid R)\cdot\P(R)\\
	&= \frac{b-1}{S - 1} \cdot \frac{n}{S}
	+ \frac{b}{S - 1} \cdot \frac{w}{S} + \frac{b}{S - 1} \cdot \frac{r}{S}\\
	&= \frac{b(b+w+r) - b}{S(S-1)} = \frac{b(S-1)}{S(S-1)} = \frac{b}{S}
	\end{align*}
\end{solution} 

\question
Let $\{B_1,\dots,B_n\}$ be a partition of $\Omega$ such that $\P(B_i) > 0$ for all $i$.
Then for every event $A$ such that $\P(A) > 0$ we have
\[
\P(B_i \mid A)
= \frac{\P(A \mid B_i)\P(B_i)}{\sum_{j = 1}^n \P(A \mid B_j) \P(B_j)}
\]

\begin{solution}
	$\P(B_i \mid A) \P(A) = \P(A \cap B_i) = \P(A \mid B_i)\P(B_i)$ by the definition of conditional probability.
	Hence
	\[
	\P(B_i \mid A) = \frac{\P(A \mid B_i)\P(B_i)}{\P(A)}
	\]
	
	Now, according to the formula of total probabilities, we have
	$\P(A) = \sum_{i = 1}^n \P(A \mid B_i) \P(B_i)$.
\end{solution} 


\question
We consider two six-sided dice, one is balanced, the other is rigged(loaded dice).
We denote $ p_i $ the probability that the rigged die falls on the face  $i$ ($i\in\{1,2,3,4,5,6\}$).
\begin{enumerate}
	\item Describe the probability space.
	
	\begin{solution}
		The sample space,:
		\[
		\Omega = \{1,2,3,4,5,6\} ×\{1,2,3,4,5,6\}
		\]
		The set of events $\mathcal P(\Omega)$ because $\Omega$ is countable.
		the probability $P$ est:
		\[
		\forall (i,j) \in \Omega,\;  P(i,j) = \frac{p_j}{6}
		\]
		With $p_j$ the probability that the rigged  one returns the number $j$.
		We also have equality $p_1+p_2+p_3+p_4+p_5+p_6 = 1$.
	\end{solution}
	
	\item
	\begin{enumerate}
		\item What is the probability of rolling a double ?
		
		\begin{solution}
			The probability of rolling a double is the sum of the probabilities of each double:
			\[
			\sum_{i=1}^6 Pr(i,i) = \frac{1}{6} \sum_{i=1}^6 p_i = \frac{1}{6}
			\]
		\end{solution}
		
		\item What is the probability that the sum of the dice is equal to $7$~?
		
		\begin{solution}
			The probability of making the sum of the dice equal to 7 is:
			\[
			\sum_{i=1}^6 Pr(i,7-i) = \frac{1}{6} \sum_{i=6}^1 p_i
			= \frac{1}{6}
			\]
		\end{solution}
	\end{enumerate}
\end{enumerate}


%




%\question
%Show that a finite monoid is a quotient of a free monoid.
%
%\begin{solution}
%	Let $\Sigma$ be an alphabet with a bijection($\phi$) to $M$. then the monoid morphism $\hat\phi$ that extends $\phi$ is subjective.
%	
%	\[
%	\begin{array}{ccc}
%	\Sigma & \hspace{-0.5em}\xrightarrow{\phi} & M \\
%	\hspace{2.0em}\searrow & & \hspace{-1.0em}\nearrow_{\hat\phi} \\
%	& \hspace{-0.5em}\Sigma^* &
%	\end{array}
%	\]
%\end{solution}
%
%
% 
%
%  \titledquestion{Groupes libres}
%  Soit $\Sigma$ un alphabet fini.
%  On note $\bar{\Sigma}$ une copie de $\Sigma$~;
%  $\bar{\Sigma} = \{ \bar{a} \mid a \in \Sigma\}$.
%  Pour chaque lettre $a \in \Sigma$, on note $\bar{\bar{a}} = a$.
%  L'application $x \to \bar{x}$ ainsi définit une involution de 
%  $\Sigma \sqcup \bar{\Sigma}$ qui échange $\Sigma $ et $\bar{\Sigma}$.
%
%  On note $L$ le monoïde libre sur l'alphabet $\Sigma \sqcup \bar{\Sigma}$.
%
%  On appelle \textit{opération élémentaire} sur un mot
%  $w=u_1u_2...u_p, u_i \in \Sigma \sqcup \bar{\Sigma}$~:
%  \begin{itemize}
%    \item{\textit{Une insertion} :}
%      $u_1u_2..u_i\, u\bar{u}\, u_{i+1}...u_p$ pour un $i$ entre $0$ et $p$ et
%      $u \in \Sigma \sqcup \bar{\Sigma}$.
%    \item{\textit{Une suppression} :}
%      $u_1u_2..u_{i-1}u_{i+2}...u_p$ pour un $i$ entre $1$ et $p-1$
%      tel que $u_{i+1} = \bar{u_i}$.
%  \end{itemize}
%
%  \begin{enumerate}
%    \item On définit sur $L$ une relation en posant $w \sim w'$ s'il existe une
%      suite finie de mots $w_1=w,w_2,...,w_{n-1}, w_n=w'$ tels que $w_{i+1}$
%      est obtenu à partir de $w_i$ par une opération élémentaire.
%
%      Démontrer que $\sim$ est une congruence.
%
%    \item On dit qu'un mot $w$ est \textit{réduit} si on ne peut pas faire de
%      suppression dans $w$.
%      \begin{enumerate}
%        \item Démontrer que toute classe de congruence contient un mot réduit.
%
%          \begin{solution}
%            Un mot de longueur minimale dans une classe de congruence est réduit.
%          \end{solution}
%
%        \item On se propose de justifier que toute classe de congruence
%          contient un unique mot réduit.
%          Soit $w$ et $w'$ deux mots réduits congruents.
%          Soit $w_1=w,w_2,...,w_{n-1}, w_n=w'$ tels que $w_{i+1} $ est obtenu à
%          partir de $w_i$ par une opération élémentaire et tels que
%          $\sum_{i} |w_i|$ est minimal parmi les suites finies de mots
%          vérifiant cette propriété.
%          On suppose $w\neq w'$ donc $n>1$.
%          \begin{enumerate}
%            \item Justifier que $|w| < |w_2| $ et $|w'| < |w_{n-1}|$.
%
%              \begin{solution}
%                Comme $w$ est réduit, la suite commence par une insertion.
%                De même, comme $w'$ est réduit, la suite finit par une
%                suppression.
%              \end{solution}
%
%            \item En déduire qu'il existe $i$ tel que 
%              % $|w_i| > |w_{i-1}|$ et $|w_i| > |w_{i+1}|$,
%              $w_i$ obtenu à partir de $w_{i-1}$ à partir d'une insertion et 
%              $w_{i+1} $ est obtenu à partir de $w_i$ à partir d'une suppression.
%
%              \begin{solution}
%                Comme $|w|<|w_2|$ et $|w'|<|w_{n-1}|$, il existe $i$ tel que
%                $|w_i|>|w_{i-1}|$ et $|w_i|>|w_{i+1}|$.
%              \end{solution}
%
%            \item Soit $a, b \in \Sigma \sqcup \bar{\Sigma}$ et $s,t$ tels que~:
%              $w_{i-1} = u_1u_2..u_p$, 
%              $w_i=u_1u_2..u_sa\bar{a}u_{s+1}...u_p = v_1....v_{p+2}$
%              et $w_{i+1} = v_1...v_{t-1}v_{t+1}...v_{p+2}$
%              avec $v_t=b$ et $V_{t+1}=\bar{b}$.
%              En étudiant les cas où ces deux opérations se chevauchent ou non,
%              aboutir à une contradiction.
%
%              \begin{solution}
%                Si les deux opérations ne se chevauchent pas, on aurait pu
%                commencer par la suppression puis effectuer l'insertion.
%                Dans ce cas, la suite $w_1=w, w_2, \dots,$
%                $w_{i-1}, w_i', w_{i+1}, \dots, w_{n-1}, w_n=w'$ représenterait
%                une suite d'opérations élémentaires avec $|w_i'|=|w_i|-4$, ce
%                que contredit la minimalité de $\sum_i |w_i|$.
%
%                Si les deux opérations de font au même endroit, alors
%                $w_{i-1}=w_{i+1}$, on peut supprimer $w_i$ et $w_{i+1}$ de la
%                suite
%                $w_1=w, w_2, \dots, w_{i-1}, w_i', w_{i+1}, \dots, w_{n-1},$
%                $w_n=w'$, ce que contredit la minimalité de $\sum_i |w_i|$.
%
%                Si les deux opérations se chevauchent sur une lettre~: \\
%                $w_i = u_1 u_2 \dots u_s a \bar{a} u_{s+1} \dots u_p$,
%                $u_{s+1}=a$ et $w_{i+1}=u_1 u_2 \dots u_s a u_{s+2} \dots u_p$.
%                Mais alors
%                $w_{i-1} = u_1 u_2 \dots u_s a u_{s+2} \dots u_p = w_{i+1}$
%                et à nouveau on peut supprimer $w_i$ et $w_{i+1}$ de la suite.
%                De même, si $u_s=\bar{a}$ et $\\
%                w_{i+1}=u_1 u_2 \dots u_{s-1} \bar{a} u_{s+2} \dots u_p =
%                w_{i-1}$.
%              \end{solution}
%          \end{enumerate}
%      \end{enumerate}
%    \item On note $GF$ le monoïde $ L/\sim$ et $\pi$ la surjection canonique de
%      $L$ sur $GF$.
%      \begin{enumerate}
%        \item Démontrer que $\pi$ injecte $\Sigma$ dans $GF$.
%
%          \begin{solution}
%            Une lettre est un mot réduit.
%          \end{solution}
%
%        \item Démontrer que $GF$ est un groupe engendré par $\pi(\Sigma)$.
%
%          \begin{solution}
%            On remarque que $u\bar{u} \sim \varepsilon \bar{u}u$, pour tout
%            $u \in \Sigma \sqcup \bar{\Sigma}$, donc
%            $\pi(u)\pi(\bar{u})=\pi(\bar{u})\pi(u)=1$, les éléments de
%            $\pi(\Sigma)$ sont tous inversibles.
%
%            Si $w=u_1 \dots u_n \in L$, $u_i \in \Sigma \sqcup \bar{\Sigma}$,
%            on a
%            $\pi(w)=\pi(u_1) \cdots \pi(u_n) \in \langle \pi(\Sigma) \rangle$.
%            De plus, $\pi(w)\pi(u_n)^{-1} \cdots \pi(u_1)^{-1}=1$ et 
%            $\pi(u_n)^{-1} \cdots \pi(u_1)^{-1}\pi(w)=1$
%          \end{solution}
%
%        \item Quel est ce groupe lorsque $\Sigma$ est un singleton~? 
%
%          \begin{solution}
%            $\Z$.
%          \end{solution}
%      \end{enumerate}
%    \item Soit $\phi$ une application de l'ensemble $\Sigma$ dans un groupe
%      $G$.
%      On étend $\phi$ sur $\bar{\Sigma}$ en posant
%      $phi(\bar{u}) = \phi(u)^{-1}$, pour tout $u$ dans $\Sigma$.
%      Démontrer qu'il existe un unique morphisme de groupes de $GF$ dans $G$
%      prolongeant $\phi$.
%
%      \begin{solution}
%        On sait déjà qu'il existe un morphisme de monoïdes de $L$ dans $G$ qui
%        prolonge $\phi$.
%        Notons $\hat\phi$ ce morphisme.
%        \[
%          \begin{array}{ccc}
%            \Sigma \sqcup \bar\Sigma & \xrightarrow{\phi} & G \\
%            \hspace{1.0em}\searrow & & \hspace{-1.0em}\nearrow_{\hat\phi} \\
%            & L &
%          \end{array}
%        \]
%
%        Or on vérifie qu'en passant d'un mot $w$ à un mot $w'$ par une
%        opération élémentaire, $\hat\phi(w)=\hat\phi(w')$, donc deux mots
%        congruents ont une même image par $\hat\phi$.
%        Donc $\hat\phi$ passe au quotient.
%        On note $\tilde\phi$ l'application ainsi obtenue de $GF$ dans $G$.
%        \[
%          \begin{array}{cccc}
%            \Sigma \sqcup \bar\Sigma & \xrightarrow{\phi} & & G \\
%            \hspace{1.0em}\searrow & & \nearrow_{\hat\phi} &
%            \uparrow \tilde\phi \\
%            & L & \xrightarrow{\pi} & GF
%          \end{array}
%        \]
%        
%        Comme $\tilde\phi(\pi(a))=\phi(a)$ pour tout $a \in \Sigma$ et que
%        $GF = \langle \pi(\Sigma) \rangle$, on a l'unicité.
%      \end{solution}
%
%    \item On note $L_R$ l'ensemble des mots réduits.
%      \begin{enumerate}
%        \item Démontrer que tout facteur d'un mot réduit est réduit.
%        \item Soit $u \in \Sigma$.
%            % Pour $w\in L_R$, $w=u_1...u_p$, avec $u_i \in \Sigma \cupsquare
%            % \bar{\Sigma}$,
%            % démontrer que $wu \in L_R$ si et seulement si $u_p \neq \bar{u}$.
%          Justifier qu'on peut définir une application $\sigma_u$ de $L_R$ dans
%          lui-même en posant~:
%          \[
%            \sigma_u : w \to \left\{
%              \begin{array}{cl}
%                uw & \text{ si } uw \in L_R, \\
%                v & \text{ si } w=\bar{u}v.
%              \end{array}
%            \right.
%          \]
%
%          \begin{solution}
%            Soit $w \in L_R$, $w = u_1 \dots u_p$,
%            $u_i \in \Sigma \sqcup \bar\Sigma$.
%
%            $v = u_2 \dots u_p$ est réduit car suffixe de $w$, donc si
%            $u_1=\bar{u}$, $w=\bar{u}v$ avec $v \in L_R$.
%            Sinon, $u_1 \neq \bar{u}$, donc $uw$ est réduit car comme $w$ est
%            réduit, la seule suppression à envisager aurait été $u u_1$.
%            On a donc bien défini une application de $L_R$ dans $L_R$.
%          \end{solution}
%
%        \item Démontrer que $\sigma_u$ est une permutation de $L_R$.
%
%          \begin{solution}
%            Par définition, on a $\sigma_{\bar{u}} \circ \sigma_u = Id$ et 
%            $\sigma_u \circ \sigma_{\bar{u}} = Id$, donc $\sigma_u$ est une
%            permutation.
%          \end{solution}
%
%        \item Soit $\sigma : \Sigma \rightarrow L$ l'application telle que
%          $\sigma(u) = \sigma_u$.
%          On note $\hat{\sigma}$ le morphisme de groupes prolongement de
%          $\sigma$ de $L$ dans $\frak{S}(L_R)$.
%          Si $w \in L_R$, démontrer que $\sigma_w(\varepsilon)= w$.
%
%          \begin{solution}
%            Soit $w = u_1 \dots u_p$ un mot réduit.
%            Justifions par récurrence sur $p$ que
%            $\hat\sigma_w(\varepsilon) = w$~: 
%            $\hat\sigma_w = \hat\sigma_{u_1}\hat\sigma_{u_2 \dots u_p} =
%            \sigma_{u_1}\hat\sigma_{u_2 \dots u_p}$.
%            Par hypothèse de récurrence,
%            $v=\hat\sigma_{u_2 \dots u_p}(\varepsilon) = u_2 \dots u_p$, donc
%            $\hat\sigma_w(\varepsilon)=\sigma_{u_1}(v)$.
%            Comme $w$ est réduit, $u_1 \neq \bar{u_2}$, donc
%            $\sigma_{u_1}(v) u_1 v = w$.
%          \end{solution}
%
%        \item Retrouver ainsi l'unicité du mot réduit dans une classe de
%          congruence.
%
%          \begin{solution}
%            On note $\tilde\sigma$ le morphisme de groupes prolongement de
%            $\sigma$ de $GF$ dans $\mathfrak{S}(L_R)$.
%            Soit $w,w'$ deux mots congruents, ils définissent un même élément
%            dans $GF$, $\hat\sigma(w) = \tilde\sigma(\pi(w)) =
%            \tilde\sigma(\pi(w')) = \hat\sigma(w')$.
%            Donc
%            $w = \hat\sigma(w)(\varepsilon) = \hat\sigma(w')(\varepsilon) = w'$.
%          \end{solution}
%      \end{enumerate}
%  \end{enumerate}
\end{questions}
\end{document}

% vim: spell spelllang=fr
